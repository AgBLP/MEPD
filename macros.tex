%%%% header
% \pagestyle{fancy}
% \lhead { % left header
%   \textbf{\footnotesize \'Ecole normale supérieure}
%   \newline
%   \footnotesize Préparation à l'agrégation de physique-chimie option physique
% }
% \chead{ % central header
% }
% \rhead{ % right header
%   \hfill \textbf{\footnotesize Compte-rendu de leçon de chimie}
%   \newline \hfill
%   \footnotesize 2021-2022
% }
% \renewcommand{\headrulewidth}{0.4pt}


%%%%%%%%%%%%%%%%%%%%%%%%%%%%%%%%%%%%%%%%%%%%%%%%%%%%%%%%%%%%
%%%% blocks for higlight
\newenvironment{highlightBlock}[1]{%
  \begin{tcolorbox}
  [
    breakable, enhanced jigsaw, % to break box over page
    arc = 4mm, % curvature
    title = \textbf{#1}, % title
    coltitle = white, % title font color
    colbacktitle = Prune, % light-gray title
    colback = white, % white background
    colframe = Prune % dark frame
  ]
}
{
  \end{tcolorbox}
}


%%%%%%%%%%%%%%%%%%%%%%%%%%%%%%%%%%%%%%%%%%%%%%%%%%%%%%%%%%%%
%%%% ball of colour with text
\newcommand\ballText[2]{%
  \shorthandoff{;}
  \tikz \node[fill, circle, color=#2,
              inner sep=0pt, text width=20pt,
              font=\footnotesize, align=center]
        {\textbf{\color{white} #1}};
  \shorthandon{;}
}
\newcommand\cyanBallText[1]{\ballText{#1}{Prune}}

%%%% rectangle of colour with text
\newcommand\rectText[3]{%
  \begin{tcolorbox}
  [
    arc = 1mm, % curvature
    colback = #3, % box color
    colframe = #3, % box color,
    width = 48pt,
    height = 18pt,
    %bean arc,
    halign = center,
    valign = center,
    after
  ]
    \textcolor {#2} {\textbf{#1}}
  \end{tcolorbox}
}
\newcommand\cyanRectText[1]{\rectText{#1}{white}{Prune}}

%%%% rectangle of colour
\newcommand\rect[3]{%
  \shorthandoff{;}
  \tikz \node (rect) [draw, fill, color=#1,
              minimum width=#2,
              minimum height=#3] {};
  \shorthandon{;}
}
\newcommand\cyanRect[2]{\rect{Prune}{#1}{#2}}


%%%%%%%%%%%%%%%%%%%%%%%%%%%%%%%%%%%%%%%%%%%%%%%%%%%%%%%%%%%%
%%%% cv entry
\newcommand{\cvEntry}[2]{
  \noindent
  \rect{#1}{80pt}{2pt}
  \textcolor{#1} {
    \textsc {\textbf {#2}}
  }
}
%%%% paragraph
\newcommand{\sectionPara}[2]{
  \vspace{10pt}
  \noindent
  \rect{#1}{50pt}{2pt}
  \textbf {#2}
}


%%%%%%%%%%%%%%%%%%%%%%%%%%%%%%%%%%%%%%%%%%%%%%%%%%%%%%%%%%%%
%%%% For aligned enumeration (cv entry)
\newenvironment{alignedColumn}{%
  \begin{longtable} { l p{0.8\textwidth} }
}
{
  \end{longtable}
}


%%%%%%%%%%%%%%%%%%%%%%%%%%%%%%%%%%%%%%%%%%%%%%%%%%%%%%%%%%%%
%%%% Section
\newcommand{\boxSection}[1]{%
  \refstepcounter{section}
  % number and text
  \cyanRectText{
    \textbf {\large \arabic{section}}
  }
  
  \vspace{-18pt}
  \hspace{-16pt}
  \textbf{\Large \phantom{right} #1}
  
  % subsection counter update
  \setcounter{subsection}{0}
  \addcontentsline{toc}{section}{\protect\numberline{} #1}
}%



%%%%%%%%%%%%%%%%%%%%%%%%%%%%%%%%%%%%%%%%%%%%%%%%%%%%%%%%%%%%
%%%% Background image
%%%% (1,2): position ; 
%%%% 3: width ; 
%%%% 4: image_name ;
%%%% 5: caption ;
\newcommand{\backgroundImage}[5]{% 
  \begin{tikzpicture}[remember picture, overlay] 
    \node[anchor = north west, inner sep = 0pt] (image)
    at ($ (current page.north west) + (#1 mm, -#2 mm) $) {
      \includegraphics[width=#3] {#4}
    };
    \node[text width=#3, align = center, below = of image] {
      #5
    };
  \end{tikzpicture}
}