\boxSection{Conclusion générale}
\vspace{2mm}
\`{A} travers mon travail de thèse, j'ai pu mener à bien un projet de recherche à la pointe des questionnements fondamentaux de matière condensée. J'ai étudié l'évolution des propriétés magnétiques de la barlowite lorsqu'on substitue progressivement une partie de ses cuivres interplans par du zinc dans le but de découpler magnétiquement les plans kagomé et espérer réaliser un liquide de spins quantique. Mes travaux ont permis la compréhension des spectres RMN dans ces matériaux et la mise en place d'une méthode fiable pour isoler la contribution intrinsèque des plans kagomé, celle-ci étant sujette à d'importants débats dans la communauté scientifique du magnétisme frustré. \\

Il m'a tenu à c\oe ur d'effectuer une mission d'enseignement à la faculté des sciences d'Orsay en complémentarité de ce travail de recherche. Cette expérience a été déterminante dans mon choix de préparer pendant un an le concours de l'agrégation externe spéciale au centre de préparation de Montrouge. Cette formation m'a permis de réinvestir des connaissances et des compétences apprises en thèse afin de prendre du recul sur la physique et la chimie dans la perspective de mon futur métier de professeur dans ces disciplines.\\

Mon travail de thèse revêt d'une dimension expérimentale importante que j'ai souhaité mettre en valeur dans ce dossier. Devant mes futurs élèves, je souhaiterai d'autant plus insister sur l'aspect expérimental de la physique et de la chimie, qui est parfois nécessaire et cruciale pour, d'une part vérifier ou explorer des modèles théoriques, et d'autre part comparer à d'autres études publiées dans la littérature. Le concept de démarche scientifique illustré par la mise en \oe uvre de protocoles expérimentaux est une méthode que je souhaiterai transmettre à mes futurs élèves.
%
%Le principe du cryostat peut être une belle illustration de l'utilisation des principes de la thermodynamique et de la caractérisation des échanges thermiques pouvant être vues en 2ème année de CPGE.