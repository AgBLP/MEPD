\boxSection{Formation, expériences et implications personnelles au regard du métier d'enseignant}
\vspace{2mm}

Tout au long de mon parcours académique mais aussi personnel, j'ai confirmé et développé un enthousiasme instinctif pour le partage et la transmission des savoirs scientifiques et de la vie citoyenne. Je développe dans cette partie les liens entre ma formation scientifique, mes expériences professionnelles et personnelles au regard des compétences attendues d'un professeur agrégé de Physique-Chimie.
\subsection{Expériences professionnelles}
Cet enthousiasme prend ses racines dans mes expériences en tant qu'animateur en centre de loisirs et en colonies de vacances. Ces cadres éducatifs non scolaires ont été des lieux et des moments pour, d'une part, écouter et accompagner des enfants et des adolescents dans leur développement personnel, et d'autre part, générer de la curiosité et faire découvrir le monde complexe dans lequel nous vivons. Par exemple, pour introduire le système solaire à des enfants de 6 à 11 ans en centre de loisirs, j'avais créé une semaine thématique lors de laquelle j'ai mené différentes activités manuelles (création d'une fresque du système solaire) et sportives (entrainement d'un astronaute) en essayant d'introduire quelques notions que je connaissais sur le sujet, de manière didactique à travers ces activités. La formation au BAFA\footnote{Brevet d'Aptitude aux Fonctions d'Animateur} et ces premières expériences d'encadrement m'ont appris l'importance de construire un cadre d'activité, d'en donner les règles et les limites à un public jeune, notamment face à des adolescents qui testent régulièrement ces limites. En tant que futur professeur, je ferai par exemple attention à ce que les règles de vie en classe, en particulier en TP de physique et chimie, soient bien établies dès les premières séances de l'année et régulièrement rappelées. %Mes expériences en colonies de vacances m'ont également permises de me rendre compte de l'importance du dialogue et du sens de la coopération au sein d'une équipe d'encadrement. J'ai pu me rendre compte qu'une bonne coordination et une bonne ambiance au sein d'une équipe se reflètent également au sein des groupes que j'ai pu encadrer. 
\\

Mon intérêt particulier pour l'enseignement des sciences et l'accompagnement des élèves dans leur cursus s'est développé au cours de ma formation universitaire. Lors de mes deux premières années au Magistère de Physique Fondamentale d'Orsay, j'ai pu transmettre et consolider des savoirs scientifiques fondamentaux à travers du tutorat d'élèves en difficulté de première année de licence, par petits groupes de deux ou trois élèves. En discutant avec les élèves au cours des séances, je me suis aperçu que la plupart travaillaient en dehors de leurs études, souvent le week-end, ce qui pouvait expliquer en partie leurs difficultés scolaires. Les problématiques ne sont pas les mêmes dans le secondaire (plutôt des problématiques sociales, familiales ou liées à l'adolescence) mais j'ai pu ainsi comprendre l'importance d'être à l'écoute des élèves au cours de l'année et de les suivre individuellement. En complément de ce suivi, j'envisagerai également, dans un futur rôle de professeur principal, des séances de vie de classe où les élèves pourront remonter et partager collectivement leurs difficultés éventuelles sur le déroulement des enseignements et de leur organisation. Ils pourront ainsi soumettre leurs ajustements au conseil de classe de fin de trimestre par l'intermédiaire de leurs délégués.\\

Motivé par cette première expérience d'enseignement, j'ai eu la chance de pouvoir l'enrichir pendant mon doctorat par un monitorat de trois ans à la faculté des sciences de l'université Paris-Saclay. Cette expérience a confirmé ma vocation de professeur. J'y ai pu appréhender la gestion d'une classe ainsi que la gestion du temps d'une séance et d'un planning de suivi de séances, celui-ci étant contraint par l'avancement du programme et la perspective des examens. J'ai trouvé particulièrement intéressant les séances de TP par demi-groupes et en binômes qui permettaient aux étudiants de se poser entre eux des questions conceptuelles (\og finalement, elle est réelle ou virtuelle l'image à travers un miroir ? \fg) et pratiques dans un temps imparti avec une contrainte de réalisation d'un compte rendu par binôme qui était évalué. Je trouverai intéressant de mettre ponctuellement en place des TP \og moins cadrés \fg ~dans lesquels je donnerai un objectif à atteindre (par exemple : \og démontrer expérimentalement la loi de Descartes sur la réfraction \fg ~) et je laisserai aux élèves la liberté de concevoir et valider un protocole expérimental à l'aide du matériel que j'aurai préalablement défini. Ce sera une occasion pour eux d'être confrontés à la mise en \oe uvre (accompagnée) d'une démarche scientifique. Cette séance pourra être également associée à une séance de recherche documentaire au CDI en lien avec le/la documentaliste.\\

Mon expérience a été marquée par la période de confinement liée à la pandémie de la COVID qui a montré à quel point l'enseignement en distanciel pouvait poser des difficultés. Certains élèves m'avaient fait part de leur détresse liée aux différents confinements. Me rendre disponible et à l'écoute a pu les rassurer en partie et m'a permis de faire remonter l'information auprès des responsables de la licence. J'ai pu ainsi orienter ces étudiants vers le personnel de l'université qualifié pour les prendre en charge. Je continuerai à garder cette attitude dans mon futur métier de professeur. Cette expérience m'a également appris à adapter mon enseignement en utilisant des nouveaux outils informatiques : plateforme de visioconférence, utilisation de tableau blanc interactif pour assurer les TD. Au cours de cette période, j'ai aussi pu découvrir des méthodes d'enseignement innovantes à travers les conférences et expériences confinées du chercheur et vulgarisateur Julien Bobroff \footnote{\url{https://www.youtube.com/results?search_query=julien+bobroff+conférence+confinée}}. Certains pourraient me servir de supports didactiques pour des futures expériences à destination des élèves.\\

En plus de la transmission des savoirs fondamentaux en TD ou en TP, j'ai eu la possibilité d'encadrer des stages de recherche à différents niveaux et sur des périodes plus ou moins longues. De façon évidente, je n'ai pas présenté mon activité de recherche ni abordé les sujets de la même façon suivant l'âge et le niveau des étudiants. Par exemple, pour présenter mon activité de recherche à des élèves de troisième lors du stage de découverte dans un milieu professionnel, j'ai choisi de leur montrer quelques expériences visuelles sur les supraconducteurs (voir la figure \ref{fig:SC}) ou de leur montrer l'imposant liquéfacteur d'hélium présent au laboratoire qui permettait d'appréhender notre thématique de recherche et notre travail expérimental quotidien. J'ai également apprécié co-encadrer trois étudiants (un en licence 3, un en master 1 et une en master 2) lors de leur stage de recherche de fin d'année. La difficulté principale résidait en la définition d'un projet expérimental réalisable dans un temps court (de 6 semaines en licence à 3 mois en master) vis-à-vis des durées de campagnes expérimentales (plusieurs mois en général pour des expériences de RMN, voir la partie 3.2 de ce dossier). %Nous faisions des réunions hebdomadaires informelles de façon à préciser l'orientation du projet en fonction de l'avancement de l'étudiant, de la difficulté et de la longueur des mesures. Cette réflexion se menait en étroite collaboration avec les membres de l'équipe, en particulier avec mes encadrants de thèse. Quotidiennement, la nature des travaux que nous fournissions pouvait être . 
Des travaux de nature bibliographique, expérimentale ou de travail de synthèse pouvaient être réalisés en autonomie afin d'impliquer et responsabiliser au mieux ces étudiants dans leur projet. Pour un de ces étudiants, il a été très difficile de le laisser autonome sur son projet et j'ai dû compenser moi-même un manque de travail. Avec le recul sur cette expérience, j'aurais adapté différemment son stage en lui demandant des synthèses écrites sur le déroulé de son travail et de sa compréhension vis-à-vis de celui-ci par exemple. Ces premières expériences de suivi de projets me semblent être formatrices et me seront utiles pour appréhender la préparation et l'évaluation du grand oral du baccalauréat ou dans le suivi des projets de TIPE de CPGE par exemple.

%J'ai également participé à l'animation d'un stand sur la supraconductivité à destination du grand public dans le cadre de la Fête de la Science à l'UPSaclay. Cette expérience m'a entrainé à adapter mon discours face à un public non nécessairement scientifique.
\subsection{Représentant des non-permanents}
Au-delà de ma propre thématique de recherche, j'ai bénéficié d'un environnement scientifique très riche au sein du Laboratoire de Physique des Solides (LPS) d'Orsay. En prenant le rôle d'un des trois représentants des non-permanents de mon laboratoire, j'ai souhaité porter et participer à des projets de rassemblements et d'animations scientifiques que je trouvais très complémentaire à mon travail de thèse. J'ai organisé par exemple la "journée des non-permanents" de mon laboratoire qui permettait de valoriser les travaux de recherche d'une trentaine de doctorants et post-doctorants de tous les domaines de la physique représentés au laboratoire.\\

Suite aux différentes périodes de confinement, la direction du laboratoire nous a permis de créer et réaliser un séminaire scientifique "Oxy'jeunes" à destination des doctorants et post-doctorants du LPS et du LPTMS, un laboratoire voisin. Ce séminaire avait pour but de développer les échanges scientifiques pluridisciplinaires dans un cadre naturel agréable (l'école de Physique des Houches dans les Alpes) après de longs mois de confinements préjudiciables aussi à la qualité des échanges scientifiques. J'ai pu coordonner l'organisation de ce séminaire, gérer le budget associé à ce projet et réaliser un bilan que j'ai présenté lors du conseil de laboratoire. \`{A} travers mon rôle de représentant des non-permanents, j'ai développé des compétences transverses qui me seront très utiles par exemple dans la participation à la vie d'une équipe pédagogique ou dans l'organisation de voyages scolaires thématiques qui peuvent être des rôles endossés par un professeur agrégé. \\

Je vais à présent présenter mes travaux de thèse, leur contexte et les principaux résultats obtenus. Je mettrai en perspective quelques applications associées aux concepts physiques que j'ai étudiés et utilisés pendant ma thèse dans des propositions d'activités que je pourrai être amené à mettre en place dans des cours de l'enseignement du secondaire et du supérieur.
\vspace{2mm}