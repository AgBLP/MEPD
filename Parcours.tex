\boxSection{Parcours universitaire et scientifique}


\vspace{20pt}
\cvEntry{Prune}{Parcours académique et professionnel :}

\begin{alignedColumn}
  
  \textbf{2022-2023 :} &
  Préparation à l’agrégation de physique au centre de Montrouge, \textit{ENS, Sorbonne Université, Paris-Saclay}\\
  %
  \textbf{2018-2021 :} & Thèse de physique de l'Université Paris-Saclay, intitulée \og \textbf{\PhDTitle} \fg{}, sous la direction du Pr. F. Bert, soutenue le Jeudi 02 Décembre 2021 au Laboratoire de Physique des Solides (LPS) à Orsay \\
  %
  \textbf{2017-2018 :} &
  Master 2 Concepts Fondamentaux de la Physique (ICFP), Parcours Matière Condensée, \textit{ENS, UPMC, Paris Diderot, Paris-Saclay}. \\
  %
  \textbf{2015-2018 :} & Magistère de physique fondamentale, \textit{Université Paris-Saclay} \\
  %
  \textbf{2013-2015 :} & 
Classes Préparatoires aux Grandes Ecoles - MPSI/MP, \textit{lycée La Pérouse-Kérichen à Brest} \\  %
\end{alignedColumn}


%%%% Stage
\cvEntry{Prune}{Expériences de recherche (hors doctorat) :}

\begin{alignedColumn}

  \textbf{2018 :} & \'{E}tude par RMN du fluor de la barlowite : un nouveau composé kagomé antiferromégnétique, sous la direction du Pr. F. Bert au \textit{LPS, Orsay} (stage de Master 2)\\
  %
  \textbf{2017 :} &
  Croissance cristalline, caractérisation avancée de nouveaux matériaux quantiques, sous la direction du Pr. B. Gaulin à l'\textit{Université de McMaster, Canada} (stage de Master 1)\\
  %
  \textbf{2016 :} &
  \'Etude de la destruction de la supraconductivité dans des films minces de Vanadium, sous la direction du Dr. C. Marrache-Kikuchi au \textit{CSNSM, Orsay} (stage de Licence 3) \\
  % 
\end{alignedColumn}


%%%% Monitorat
\cvEntry{Prune}{Animation, vulgarisation et expériences d'enseignement :}

\begin{alignedColumn}

  \textbf{Mars 2023 :} & Stage facultatif de 3 jours au lycée Condorcet de Montreuil (de la seconde au BTS) \\
  %
  \textbf{2018-2021 :} & Monitorat de Physique (64h/an) à l'Université Paris-Saclay : \begin{itemize}\setlength{\itemsep}{0.05cm}
  \item travaux dirigés de mécanique (L1),
  \item travaux dirigés et travaux pratiques d'optique géométrique (L1),
  \item travaux dirigés de mathématiques pour la physique (L3-magistère).
  \end{itemize} \\
  %
  \textbf{2018-2021 :} & Représentant des non-permanents au conseil du LPS \\
  %
  \textbf{2018-2021 :} &
  Animation du stand "Supraconductivité" pour la Fête de la science au LPS \\
  \textbf{2016 :} & Tutorat de Physique pour des élèves de L1 en difficulté\\
  \textbf{Depuis 2013 :} & Brevet d'Aptitude aux Fonctions d'Animateur (BAFA)
  %
\end{alignedColumn}

\newpage